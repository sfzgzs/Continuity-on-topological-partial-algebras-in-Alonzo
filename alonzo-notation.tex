%% LaTeX Macros for Alonzo Notation
%%
%% William M. Farmer
%%
%% McMaster University

%% Requires:

\usepackage{amssymb}
\usepackage{amsmath}
\usepackage{phonetic}
\usepackage[dvipsnames]{xcolor}

%%%%%%%%%%%%%%%%%%%%%%%%%%%%%%%%%%%%%%%%%%%%%%%%%%%%%%%%%%%%%%%%%

% MISCELLANEOUS MACROS

\newcommand{\mName}[1]{\mathsf{#1}}
\newcommand{\mSet}[1]{\{ #1 \}}
\newcommand{\mTuple}[1]{( #1 )}
\newcommand{\mList}[1]{[ #1 ]}
\newcommand{\mSeq}[1]{\langle #1 \rangle}
\newcommand{\mSeqlike}[1]{\mSeq{\!\mSeq{#1}\!}}
\newcommand{\mAbs}[1]{\lvert #1 \rvert}
\newcommand{\mNorm}[1]{\lVert #1 \rVert}
\newcommand{\mDot}{\mathrel{.}}

%%%%%%%%%%%%%%%%%%%%%%%%%%%%%%%%%%%%%%%%%%%%%%%%%%%%%%%%%%%%%%%%%

% FORMAL NOTATION

% Types

\newcommand{\fBoolTy}{\mName{BoolTy}}
\newcommand{\fBaseTy}[1]{\mName{BaseTy}(#1)}
\newcommand{\fFunTy}[2]{\mName{FunTy}(#1,#2)}
\newcommand{\fProdTy}[2]{\mName{ProdTy}(#1,#2)}

% Expressions

\newcommand{\fVar}[2]{\mName{Var}(#1,#2)}
\newcommand{\fCon}[2]{\mName{Con}(#1,#2)}
\newcommand{\fEq}[2]{\mName{Eq}(#1,#2)}
\newcommand{\fFunApp}[2]{\mName{FunApp}(#1,#2)}
\newcommand{\fFunAbs}[3]{\mName{FunAbs}(\fVar{#1}{#2},#3)}
\newcommand{\fDefDes}[3]{\mName{DefDes}(\fVar{#1}{#2},#3)}
\newcommand{\fOrdPair}[2]{\mName{OrdPair}(#1,#2)}

%%%%%%%%%%%%%%%%%%%%%%%%%%%%%%%%%%%%%%%%%%%%%%%%%%%%%%%%%%%%%%%%%

% COMPACT NOTATION

% Types

\newcommand{\cBoolTy}{\omicron}
\newcommand{\cB}{\cBoolTy}
\newcommand{\cBaseTy}[1]{#1}
\newcommand{\cFunTy}[2]{({#1} \rightarrow {#2})}
\newcommand{\cFunTyX}[2]{{#1} \rightarrow {#2}}
\newcommand{\cFunTyB}[3]{\cFunTy {#1} {\cFunTyX {#2} {#3}}}
\newcommand{\cFunTyBX}[3]{\cFunTyX {#1} {\cFunTyX {#2} {#3}}}
\newcommand{\cFunTyC}[4]{\cFunTy {#1} {\cFunTyX {#2} {\cFunTyX {#3} {#4}}}}
\newcommand{\cFunTyCX}[4]{\cFunTyX {#1} {\cFunTyX {#2} {\cFunTyX {#3} {#4}}}}
\newcommand{\cProdTy}[2]{({#1} \times {#2})}
\newcommand{\cProdTyX}[2]{{#1} \times {#2}}
\newcommand{\cProdTyB}[3]{\cProdTy {#1} {\cProdTyX {#2} {#3}}}
\newcommand{\cProdTyBX}[3]{\cProdTyX {#1} {\cProdTyX {#2} {#3}}}
\newcommand{\cProdTyC}[4]{\cProdTy {#1} {\cProdTyX {#2} {\cProdTyX {#3} {#4}}}}
\newcommand{\cProdTyCX}[4]{\cProdTyX {#1} {\cProdTyX {#2} {\cProdTyX {#3} {#4}}}}

% Expressions

\newcommand{\cVar}[2]{({#1} : {#2})}
\newcommand{\cVarY}[2]{#1}
\newcommand{\cCon}[2]{{#1}_{#2}}
\newcommand{\cConY}[2]{#1}
\newcommand{\cEq}[2]{({#1} = {#2})}
\newcommand{\cEqX}[2]{{#1} = {#2}}
\newcommand{\cFunApp}[2]{(#1\,#2)}
\newcommand{\cFunAppX}[2]{#1\,#2}
\newcommand{\cFunAppB}[3]{(\cFunAppX {\cFunAppX {#1} {#2}} {#3})}
\newcommand{\cFunAppBX}[3]{\cFunAppX {\cFunAppX {#1} {#2}} {#3}}
\newcommand{\cFunAppC}[4]{(\cFunAppX {\cFunAppX {\cFunAppX {#1}{#2}}{#3}}{#4})}
\newcommand{\cFunAppCX}[4]{\cFunAppX {\cFunAppX {\cFunAppX {#1}{#2}}{#3}}{#4}}
\newcommand{\cFunAbs}[3]{(\lambda\, #1 : #2 \mDot #3)}
\newcommand{\cFunAbsX}[3]{\lambda\, #1 : #2 \mDot #3}
\newcommand{\cDefDes}[3]{(\mathrm{I}\, #1 : #2 \mDot #3)}
\newcommand{\cDefDesX}[3]{\mathrm{I}\, #1 : #2 \mDot #3}
\newcommand{\cOrdPair}[2]{(#1,#2)}

% Boolean Operators

\newcommand{\cTPC}{T_\cB}
\newcommand{\cT}{\cTPC}
\newcommand{\cFPC}{F_\cB}
\newcommand{\cF}{\cFPC}
\newcommand{\cAndPC}{\wedge_{\cFunTyBX{\cB}{\cB}{\cB}}}
\newcommand{\cAnd}[2]{({#1} \wedge {#2})}
\newcommand{\cAndX}[2]{{#1} \wedge {#2}}
\newcommand{\cAndB}[3]{\cAnd {#1} {\cAndX {#2} {#3}}}
\newcommand{\cAndBX}[3]{\cAndX {#1} {\cAndX {#2} {#3}}}
\newcommand{\cAndL}[1]{(#1)}   % separator is $\And$
\newcommand{\cAndLX}[1]{#1}    % separator is $\And$
\newcommand{\cImpliesPC}{\Rightarrow_{\cFunTyBX {\cB} {\cB} {\cB}}}
\newcommand{\cImplies}[2]{({#1} \Rightarrow {#2})}
\newcommand{\cImpliesX}[2]{{#1} \Rightarrow {#2}}
\newcommand{\cNegPC}{\neg_{\cFunTyX{\cB}{\cB}}}
\newcommand{\cNeg}[1]{(\neg{#1})}
\newcommand{\cNegX}[1]{\neg{#1}}
\newcommand{\cOrPC}{\vee_{\cFunTyBX{\cB}{\cB}{\cB}}}
\newcommand{\cOr}[2]{({#1} \vee {#2})}
\newcommand{\cOrX}[2]{{#1} \vee {#2}}
\newcommand{\cOrB}[3]{\cOr {#1} {\cOrX {#2} {#3}}}
\newcommand{\cOrBX}[3]{\cOrX {#1} {\cOrX {#2} {#3}}}
\newcommand{\cOrL}[1]{(#1)}    % separator is $\Or$
\newcommand{\cOrLX}[1]{#1}     % separator is $\Or$
\newcommand{\cIfPC}[1]{\mName{if}_{\cFunTyCX{\cB}{#1}{#1}{#1}}}
\newcommand{\cIf}[3]{(#1 \mapsto #2 \mid #3)}
\newcommand{\cIfX}[3]{#1 \mapsto #2 \mid #3}

% Binary Operators

\newcommand{\cBin}[3]{({#1} \mathrel{#2} {#3})}
\newcommand{\cBinX}[3]{{#1} \mathrel{#2} {#3}}
\newcommand{\cBinB}[5]{({#1} \mathrel{#2} {#3} \mathrel{#4} {#5})}
\newcommand{\cBinBX}[5]{{#1} \mathrel{#2} {#3} \mathrel{#4} {#5}}
\newcommand{\cIff}[2]{({#1} \Leftrightarrow {#2})}
\newcommand{\cIffX}[2]{{#1} \Leftrightarrow {#2}}
\newcommand{\cNotEq}[2]{({#1} \not= {#2})}
\newcommand{\cNotEqX}[2]{{#1} \not= {#2}}

% Quantifiers

\newcommand{\cForall}[3]{(\forall\, #1 : #2 \mDot #3)}
\newcommand{\cForallX}[3]{\forall\, #1 : #2 \mDot #3}
\newcommand{\cForallB}[5]{(\forall\, #1 : #2,\, #3 : #4 \mDot #5)}
\newcommand{\cForallBX}[5]{\forall\, #1 : #2,\, #3 : #4 \mDot #5}
\newcommand{\cForallC}[7]{(\forall\, #1 : #2,\, #3 : #4,\, #5 : #6 \mDot #7)}
\newcommand{\cForallCX}[7]{\forall\, #1 : #2,\, #3 : #4,\, #5 : #6 \mDot #7}
\newcommand{\cForsome}[3]{(\exists\, #1 : #2 \mDot #3)}
\newcommand{\cForsomeX}[3]{\exists\, #1 : #2 \mDot #3}
\newcommand{\cForsomeB}[5]{(\exists\, #1 : #2,\, #3 : #4 \mDot #5)}
\newcommand{\cForsomeBX}[5]{\exists\, #1 : #2,\, #3 : #4 \mDot #5}
\newcommand{\cForsomeC}[7]{(\exists\, #1 : #2,\, #3 : #4,\, #5 : #6 \mDot #7)}
\newcommand{\cForsomeCX}[7]{\exists\, #1 : #2,\, #3 : #4,\, #5 : #6 \mDot #7}
\newcommand{\cForsomeUnique}[3]{(\exists!\, #1 : #2 \mDot #3)}
\newcommand{\cForsomeUniqueX}[3]{\exists!\, #1 : #2 \mDot #3}

% Definedness

\newcommand{\cBotPC}[1]{\bot_{#1}}
\newcommand{\cEmpFunPC}[2]{\Delta_{\cFunTyX {#1} {#2}}}
\newcommand{\cIsDef}[1]{(#1{\downarrow})}
\newcommand{\cIsDefX}[1]{#1{\downarrow}}
\newcommand{\cIsUndef}[1]{(#1{\uparrow})}
\newcommand{\cIsUndefX}[1]{#1{\uparrow}}
\newcommand{\cQuasiEq}[2]{({#1} \simeq {#2})}
\newcommand{\cQuasiEqX}[2]{{#1} \simeq {#2}}
\newcommand{\cNotQuasiEq}[2]{({#1} \not\simeq {#2})}
\newcommand{\cNotQuasiEqX}[2]{{#1} \not\simeq {#2}}

% Sets

\newcommand{\cSetTy}[1]{\mSet{#1}}
\newcommand{\cIn}[2]{({#1} \in {#2})}
\newcommand{\cInX}[2]{{#1} \in {#2}}
\newcommand{\cNotIn}[2]{({#1} \not\in {#2})}
\newcommand{\cNotInX}[2]{{#1} \not\in {#2}}
\newcommand{\cSet}[3]{\mSet{{{#1} : {#2}} \mid {#3}}}
\newcommand{\cEmpSetPC}[1]{\emptyset_{\cSetTy {#1}}}
\newcommand{\cEmpSetAltPC}[1]{\mSet{\,}_{\cSetTy {#1}}}
\newcommand{\cUnivSetPC}[1]{U_{\cSetTy {#1}}}
\newcommand{\cFinSet}[2]{\textsf{{$#1$}-{$#2$}-SET}}
\newcommand{\cFinSetL}[1]{\mSet{#1}}  % separator is ","
\newcommand{\cSubseteqPC}[1]{\subseteq_{\cFunTyBX {\cSetTy #1} {\cSetTy #1} {\cB}}}
\newcommand{\cUnionPC}[1]{\cup_{\cFunTyBX {\cSetTy #1} {\cSetTy #1} {\cSetTy #1}}}
\newcommand{\cUnion}[2]{\cBin {#1} {\cup} {#2}}
\newcommand{\cUnionX}[2]{\cBinX {#1} {\cup} {#2}}
\newcommand{\cIntersPC}[1]{\cap_{\cFunTyBX {\cSetTy #1} {\cSetTy #1} {\cSetTy #1}}}
\newcommand{\cInters}[2]{\cBin {#1} {\cap} {#2}}
\newcommand{\cIntersX}[2]{\cBinX {#1} {\cap} {#2}}
\newcommand{\cComplPC}[1]{\overline{\,\cdot\,}_{\cFunTyX {\cSetTy #1} {\cSetTy #1}}}
\newcommand{\cCompl}[1]{\overline{#1}}

% Tuples

\newcommand{\cTupleTyL}[1]{(#1)}  % separator is $\times$
\newcommand{\cTupleL}[1]{(#1)}    % separator is ","
\newcommand{\cFstPC}[2]{\mName{fst}_{\cFunTyX {\cProdTy {#1} {#2}} {#1}}}
\newcommand{\cSndPC}[2]{\mName{snd}_{\cFunTyX {\cProdTy {#1} {#2}} {#2}}}

% Functions

\newcommand{\cDomPC}[2]{\mName{dom}_{\cFunTyX {\cFunTy {#1} {#2}} {\cSetTy {#1}}}}
\newcommand{\cRanPC}[2]{\mName{ran}_{\cFunTyX {\cFunTy {#1} {#2}} {\cSetTy {#2}}}}
\newcommand{\cSubfuneqPC}[2]{\subfun_{\cFunTyBX {\cFunTy {#1} {#2}} {\cFunTy {#1} {#2}} {\cB}}}
\newcommand{\cFunCompPC}[3]{\circ_{\cFunTyBX {\cFunTy {#1} {#2}} {\cFunTy {#2} {#3}} {\cFunTy {#1} {#3}}}}
\newcommand{\cFunComp}[2]{({#1} \circ {#2})}
\newcommand{\cFunCompX}[2]{#1 \circ {#2}}
\newcommand{\cRestrictPC}[2]{\restricted_{\cFunTyBX {\cFunTy {#1} {#2}} {\cSetTy {#1}} {\cFunTy {#1} {#2}}}}
\newcommand{\cRestrict}[2]{(#1 \restricted_{#2})}
\newcommand{\cRestrictX}[2]{#1 \restricted_{#2}}

% Miscellaneous Notation

\newcommand{\cTotal}[1]{\mName{TOTAL}(#1)}
\newcommand{\cTotalB}[1]{\mName{TOTAL2}(#1)}
\newcommand{\cSurj}[1]{\mName{SURJ}(#1)}
\newcommand{\cSurjB}[1]{\mName{SURJ2}(#1)}
\newcommand{\cInj}[1]{\mName{INJ}(#1)}
\newcommand{\cBij}[1]{\mName{BIJ}(#1)}
\newcommand{\cDistinctL}[1]{\mName{DISTINCT}(#1)}  % separator is ","

% Quasitypes

\newcommand{\cFunAbsQTy}[3]{\cFunAbs {#1} {#2} {#3}}
\newcommand{\cFunAbsQTyX}[3]{\cFunAbsX {#1} {#2} {#3}}
\newcommand{\cForallQTy}[3]{\cForall {#1} {#2} {#3}}
\newcommand{\cForallQTyX}[3]{\cForallX {#1} {#2} {#3}}
\newcommand{\cForallQTyB}[5]{\cForallB {#1} {#2} {#3} {#4} {#5}}
\newcommand{\cForallQTyBX}[5]{\cForallBX {#1} {#2} {#3} {#4} {#5}}
\newcommand{\cForsomeQTy}[3]{\cForsome {#1} {#2} {#3}}
\newcommand{\cForsomeQTyX}[3]{\cForsomeX {#1} {#2} {#3}}
\newcommand{\cForsomeQTyB}[5]{\cForsomeB {#1} {#2} {#3} {#4} {#5}}
\newcommand{\cForsomeQTyBX}[5]{\cForsomeBX {#1} {#2} {#3} {#4} {#5}}
\newcommand{\cDefDesQTy}[3]{\cDefDes {#1} {#2} {#3}}
\newcommand{\cDefDesQTyX}[3]{\cDefDesX {#1} {#2} {#3}}
\newcommand{\cIsDefInQTy}[2]{({#1} \downarrow {#2})}
\newcommand{\cIsDefInQTyX}[2]{{#1} \downarrow {#2}}
\newcommand{\cIsUndefInQTy}[2]{({#1} \uparrow {#2})}
\newcommand{\cIsUndefInQTyX}[2]{{#1} \uparrow {#2}}
\newcommand{\cFunQTyPC}[2]{\rightarrow_{\cFunTyBX {\cSetTy {#1}} {\cSetTy {#2}} {\cSetTy {\cFunTyX {#1} {#2}}}}}
\newcommand{\cFunQTy}[2]{\cFunTy {#1} {#2}}
\newcommand{\cFunQTyX}[2]{\cFunTyX {#1} {#2}}
\newcommand{\cFunQTyB}[3]{\cFunTyB {#1} {#2} {#3}}
\newcommand{\cFunQTyBX}[3]{\cFunTyBX {#1} {#2} {#3}}
\newcommand{\cFunQTyC}[4]{\cFunTyC {#1} {#2} {#3} {#4}}
\newcommand{\cFunQTyCX}[4]{\cFunTyCX {#1} {#2} {#3} {#4}}
\newcommand{\cProdQTyPC}[2]{\times_{\cFunTyBX {\cSetTy {#1}} {\cSetTy {#2}} {\cSetTy {\cProdTyX {#1} {#2}}}}}
\newcommand{\cProdQTy}[2]{\cProdTy {#1} {#2}}
\newcommand{\cProdQTyX}[2]{\cProdTyX {#1} {#2}}
\newcommand{\cProdQTyB}[3]{\cProdTyB {#1} {#2} {#3}}
\newcommand{\cProdQTyBX}[3]{\cProdTyBX {#1} {#2} {#3}}
\newcommand{\cProdQTyC}[4]{\cProdTyC {#1} {#2} {#3} {#4}}
\newcommand{\cProdQTyCX}[4]{\cProdTyCX {#1} {#2} {#3} {#4}}
\newcommand{\cTotalOn}[3]{\textsf{TOTAL-ON}(#1,#2,#3)}
\newcommand{\cTotalOnB}[4]{\textsf{TOTAL-ON2}(#1,#2,#3,#4)}
\newcommand{\cSurjOn}[3]{\textsf{SURJ-ON}(#1,#2,#3)}
\newcommand{\cSurjOnB}[4]{\textsf{SURJ-ON2}(#1,#2,#3,#4)}
\newcommand{\cInjOn}[2]{\textsf{INJ-ON}(#1,#2)}
\newcommand{\cBijOn}[3]{\textsf{BIJ-ON}(#1,#2,#3)}
\newcommand{\cInf}[1]{\mName{INF}(#1)}
\newcommand{\cFin}[1]{\mName{FIN}(#1)}
\newcommand{\cCount}[1]{\mName{COUNT}(#1)}

% Dependent Quasitypes

\newcommand{\cPiTy}[2]{\cFunTyBX {\cSetTy {#1}} {\cFunTy {#1} {\cSetTy {#2}}} {\cSetTy {\cFunTyX {#1} {#2}}}}
\newcommand{\cPiPC}[2]{\Pi_{\cPiTy {#1} {#2}}}
\newcommand{\cPi}[3]{(\Pi\, #1 : #2 \mDot #3)}
\newcommand{\cPiX}[3]{\Pi\, #1 : #2 \mDot #3}
\newcommand{\cSigmaTy}[2]{\cFunTyBX {\cSetTy {#1}} {\cFunTy {#1} {\cSetTy {#2}}} {\cSetTy {\cProdTyX {#1} {#2}}}}
\newcommand{\cSigmaPC}[2]{\Sigma_{\cSigmaTy {#1} {#2}}}
\newcommand{\cSigma}[3]{(\Sigma\, #1 : #2 \mDot #3)}
\newcommand{\cSigmaX}[3]{\Sigma\, #1 : #2 \mDot #3}

% Sequences

\newcommand{\cSequencesPC}[2]{\mName{sequences}_{\cSetTy {\cFunTyX {#1} {#2}}}}
\newcommand{\cSeqQTy}[1]{\mSeqlike{#1}}
\newcommand{\cStreamsPC}[2]{\mName{streams}_{\cSetTy {\cFunTyX {#1} {#2}}}}
\newcommand{\cSeqInfQTy}[1]{\mSeq{#1}}
\newcommand{\cListsPC}[2]{\mName{lists}_{\cSetTy {\cFunTyX {#1} {#2}}}}
\newcommand{\cSeqFinQTy}[1]{\mList{#1}}
\newcommand{\cConsPC}[2]{\mName{cons}_{\cFunTyBX {#2} {\cFunTy {#1} {#2}} {\cFunTy {#1} {#2}}}}
\newcommand{\cCons}[2]{({#1} :: {#2})}
\newcommand{\cConsX}[2]{{#1} :: {#2}}
\newcommand{\cHdPC}[2]{\mName{hd}_{\cFunTyX {\cFunTy {#1} {#2}} {#2}}}
\newcommand{\cTlPC}[2]{\mName{tl}_{\cFunTyX {\cFunTy {#1} {#2}} {\cFunTy {#1} {#2}}}}
\newcommand{\cNilPC}[2]{\mName{nil}_{\cFunTyX {#1} {#2}}}
\newcommand{\cEmpListPC}[2]{{\mList{\;}}_{\cFunTyX {#1} {#2}}}
\newcommand{\cListL}[1]{\mList{#1}}  % separator is ","
\newcommand{\cLenPC}[2]{\mName{len}_{\cFunTyX {\cFunTy {#1} {#2}} {#1}}}
\newcommand{\cLen}{\mAbs}
\newcommand{\cNlistsPC}[2]{\mName{nlists}_{\cFunTyX {#1} {\cSetTy {\cFunTyX {#1} {#2}}}}}
\newcommand{\cSeqNFinQTy}[2]{\mList{#1}^{#2}}

% Real Numbers

\newcommand{\cRecip}{\recip}
\newcommand{\cFrac}{\frac}
\newcommand{\cAbs}{\mAbs}
\newcommand{\cSqrt}{\sqrt}
\newcommand{\cNorm}{\mNorm}
\newcommand{\cSum}[4]{\sum\limits_{{#1} = {#2}}^{#3} {#4}}
\newcommand{\cProd}[4]{\prod\limits_{{#1} = {#2}}^{#3} {#4}}
\newcommand{\cLim}[3]{\lim\limits_{{#1} \to {#2}} {#3}}
\newcommand{\cRightLim}[3]{\lim\limits_{{#1} \to {#2}^+} {#3}}
\newcommand{\cLeftLim}[3]{\lim\limits_{{#1} \to {#2}^-} {#3}}
\newcommand{\cLimSeq}[2]{\lim\limits_{{#1} \to \infty} {#2}}
\newcommand{\cIntegral}[4]{\int_{#1}^{#2} {#3}\,d{#4}}

%%%%%%%%%%%%%%%%%%%%%%%%%%%%%%%%%%%%%%%%%%%%%%%%%%%%%%%%%%%%%%%%%

% THEOREM ENVIRONMENTS

\newtheorem{thm}{Theorem}[section]
\newtheorem{cor}[thm]{Corollary}
\newtheorem{lem}[thm]{Lemma}
\newtheorem{prop}[thm]{Proposition}
\newtheorem{eg}[thm]{Example}
\newtheorem{rem}[thm]{Remark}
\newtheorem{df}[thm]{Definition}

\newtheorem{thydef}[thm]{Theory Definition}
\newtheorem{thyext}[thm]{Theory Extension}
\newtheorem{devdef}[thm]{Development Definition}
\newtheorem{devext}[thm]{Development Extension}
\newtheorem{thytransdef}[thm]{Theory Translation Definition}
\newtheorem{thytransext}[thm]{Theory Translation Extension}
\newtheorem{devtransdef}[thm]{Development Translation Definition}
\newtheorem{devtransext}[thm]{Development Translation Extension}
\newtheorem{deftransport}[thm]{Definition Transportation}
\newtheorem{thmtransport}[thm]{Theorem Transportation}

%%%%%%%%%%%%%%%%%%%%%%%%%%%%%%%%%%%%%%%%%%%%%%%%%%%%%%%%%%%%%%%%%

% ENVIRONMENTS

\newenvironment{theory-def}[5]
{
\color{brown}
\begin{thydef}[#1]\em
\noindent
\begin{itemize} \setlength{\itemsep}{0pt}
\item[]\hspace{-3ex}\textbf{Name:} #2
\item[]\hspace{-3ex}\textbf{Base types:} #3
\item[]\hspace{-3ex}\textbf{Constants:} #4
\item[]\hspace{-3ex}\textbf{Axioms:}
\end{itemize}
 #5
\end{thydef} 
}

\newenvironment{theory-ext}[6]
{
\color{brown}
\begin{thyext}[#1]\em
\noindent
\begin{itemize} \setlength{\itemsep}{0pt}
\item[]\hspace{-3ex}\textbf{Name:} #2
\item[]\hspace{-3ex}\textbf{Extends\ } #3
\item[]\hspace{-3ex}\textbf{New base types:} #4
\item[]\hspace{-3ex}\textbf{New constants:} #5
\item[]\hspace{-3ex}\textbf{New axioms:}
\end{itemize}
#6
\end{thyext}
}

\newenvironment{dev-def}[4]
{
\color{brown}
\begin{devdef}[#1]\em
\noindent
\begin{itemize}
\item[]\hspace{-3ex}\textbf{Name:} #2
\item[]\hspace{-3ex}\textbf{Bottom theory:} #3
\item[]\hspace{-3ex}\textbf{Definitions and theorems:}
\end{itemize}
#4
\end{devdef}
}

\newenvironment{dev-ext}[4]
{
\color{brown}
\begin{devext}[#1]\em
\noindent
\begin{itemize}
\item[]\hspace{-3ex}\textbf{Name:} #2
\item[]\hspace{-3ex}\textbf{Extends\ } #3
\item[]\hspace{-3ex}\textbf{New definitions and theorems:}
\end{itemize}
#4
\end{devext}
}

\newenvironment{theory-trans-def}[6]
{
\color{brown}
\begin{thytransdef}[#1]\em
\noindent
\begin{itemize}
\item[]\hspace{-3ex}\textbf{Name:} #2
\item[]\hspace{-3ex}\textbf{Source theory:} #3
\item[]\hspace{-3ex}\textbf{Target theory:} #4
\item[]\hspace{-3ex}\textbf{Base type mapping:}
\end{itemize}
#5
\begin{itemize}
\item[]\hspace{-3ex}\textbf{Constant mapping:}
\end{itemize}
#6
\end{thytransdef}
}

\newenvironment{theory-trans-ext}[7]
{
\color{brown}
\begin{thytransext}[#1]\em
\noindent
\begin{itemize}
\item[]\hspace{-3ex}\textbf{Name:} #2
\item[]\hspace{-3ex}\textbf{Extends\ } #3
\item[]\hspace{-3ex}\textbf{New source theory:} #4
\item[]\hspace{-3ex}\textbf{New target theory:} #5
\item[]\hspace{-3ex}\textbf{New base type mapping:}
\end{itemize}
#6
\begin{itemize}
\item[]\hspace{-3ex}\textbf{New constant mapping:}
\end{itemize}
#7
\end{thytransext}
}

\newenvironment{dev-trans-def}[6]
{
\color{brown}
\begin{devtransdef}[#1]\em
\noindent
\begin{itemize}
\item[]\hspace{-3ex}\textbf{Name:} #2
\item[]\hspace{-3ex}\textbf{Source development:} #3
\item[]\hspace{-3ex}\textbf{Target development:} #4
\item[]\hspace{-3ex}\textbf{Base type mapping:}
\end{itemize}
#5
\begin{itemize}
\item[]\hspace{-3ex}\textbf{Constant mapping:}
\end{itemize}
#6
\end{devtransdef}
}

\newenvironment{dev-trans-ext}[6]
{
\color{brown}
\begin{devtransext}[#1]\em
\noindent
\begin{itemize}
\item[]\hspace{-3ex}\textbf{Name:} #2
\item[]\hspace{-3ex}\textbf{Extends\ } #3
\item[]\hspace{-3ex}\textbf{New source development:} #4
\item[]\hspace{-3ex}\textbf{New target development:} #5
\end{itemize}
\begin{itemize}
\item[]\hspace{-3ex}\textbf{New defined constant mapping:}
\end{itemize}
#6
\end{devtransext}
}

\newenvironment{def-transport}[7]
{
\color{brown}
\begin{deftransport}[#1]\em
\noindent
\begin{itemize}
\item[]\hspace{-3ex}\textbf{Name:} #2
\item[]\hspace{-3ex}\textbf{Development morphism:} #3
\item[]\hspace{-3ex}\textbf{Definition:}
\end{itemize}
#4
\begin{itemize}
\item[]\hspace{-3ex}\textbf{Transported definition:}
\end{itemize}
#5
\begin{itemize}
\item[]\hspace{-3ex}\textbf{New target development:} #6
\item[]\hspace{-3ex}\textbf{New development translation:} #7
\end{itemize}
\end{deftransport}
}

\newenvironment{thm-transport}[6]
{
\color{brown}
\begin{thmtransport}[#1]\em
\noindent
\begin{itemize}
\item[]\hspace{-3ex}\textbf{Name:} #2
\item[]\hspace{-3ex}\textbf{Development morphism:} #3
\item[]\hspace{-3ex}\textbf{Theorem:}
\end{itemize}
#4
\begin{itemize}
\item[]\hspace{-3ex}\textbf{Transported theorem:}
\end{itemize}
#5
\begin{itemize}
\item[]\hspace{-3ex}\textbf{New target development:} #6
\end{itemize}
\end{thmtransport}
}

